% Options for packages loaded elsewhere
\PassOptionsToPackage{unicode}{hyperref}
\PassOptionsToPackage{hyphens}{url}
%
\documentclass[
]{article}
\usepackage{amsmath,amssymb}
\usepackage{iftex}
\ifPDFTeX
  \usepackage[T1]{fontenc}
  \usepackage[utf8]{inputenc}
  \usepackage{textcomp} % provide euro and other symbols
\else % if luatex or xetex
  \usepackage{unicode-math} % this also loads fontspec
  \defaultfontfeatures{Scale=MatchLowercase}
  \defaultfontfeatures[\rmfamily]{Ligatures=TeX,Scale=1}
\fi
\usepackage{lmodern}
\ifPDFTeX\else
  % xetex/luatex font selection
\fi
% Use upquote if available, for straight quotes in verbatim environments
\IfFileExists{upquote.sty}{\usepackage{upquote}}{}
\IfFileExists{microtype.sty}{% use microtype if available
  \usepackage[]{microtype}
  \UseMicrotypeSet[protrusion]{basicmath} % disable protrusion for tt fonts
}{}
\makeatletter
\@ifundefined{KOMAClassName}{% if non-KOMA class
  \IfFileExists{parskip.sty}{%
    \usepackage{parskip}
  }{% else
    \setlength{\parindent}{0pt}
    \setlength{\parskip}{6pt plus 2pt minus 1pt}}
}{% if KOMA class
  \KOMAoptions{parskip=half}}
\makeatother
\usepackage{xcolor}
\usepackage{graphicx}
\makeatletter
\def\maxwidth{\ifdim\Gin@nat@width>\linewidth\linewidth\else\Gin@nat@width\fi}
\def\maxheight{\ifdim\Gin@nat@height>\textheight\textheight\else\Gin@nat@height\fi}
\makeatother
% Scale images if necessary, so that they will not overflow the page
% margins by default, and it is still possible to overwrite the defaults
% using explicit options in \includegraphics[width, height, ...]{}
\setkeys{Gin}{width=\maxwidth,height=\maxheight,keepaspectratio}
% Set default figure placement to htbp
\makeatletter
\def\fps@figure{htbp}
\makeatother
\ifLuaTeX
  \usepackage{luacolor}
  \usepackage[soul]{lua-ul}
\else
  \usepackage{soul}
\fi
\setlength{\emergencystretch}{3em} % prevent overfull lines
\providecommand{\tightlist}{%
  \setlength{\itemsep}{0pt}\setlength{\parskip}{0pt}}
\setcounter{secnumdepth}{-\maxdimen} % remove section numbering
\ifLuaTeX
  \usepackage{selnolig}  % disable illegal ligatures
\fi
\IfFileExists{bookmark.sty}{\usepackage{bookmark}}{\usepackage{hyperref}}
\IfFileExists{xurl.sty}{\usepackage{xurl}}{} % add URL line breaks if available
\urlstyle{same}
\hypersetup{
  pdftitle={6. Ablauf Asylverfahren},
  hidelinks,
  pdfcreator={LaTeX via pandoc}}

\title{6. Ablauf Asylverfahren}
\author{}
\date{}

\begin{document}
\maketitle

up:: Sources MOC\\
tags:: \#IMB/Rechtsgrundlagen \#Edelmann\\
dates:: 2023-06-06

\hypertarget{ablauf-asylverfahren}{%
\section{Ablauf Asylverfahren}\label{ablauf-asylverfahren}}

\hypertarget{einreise-von-ausluxe4ndern}{%
\subsection{Einreise von Ausländern}\label{einreise-von-ausluxe4ndern}}

\begin{itemize}
\tightlist
\item
  Deutsches Asyl kann \textbf{nur} in DE beantragt werden, d.h.
  Asylbewerber müssen einen Pass + Aufenthaltstitel besitzen (z.B.
  Visum)

  \begin{itemize}
  \tightlist
  \item
    gültiger Pass ist Pflicht
  \item
    es gibt Vereinbarungen → aus bestimmten Ländern Einreise ohne Visum
    für max. 90 Tage

    \begin{itemize}
    \tightlist
    \item
      für die Einreise: Reisepass + Auslandskrankenversicherung, bes.
      wegen medizinischen Gründen (normalerweise in Voraus bezahlt und
      für sehr kurze Zeit gültig)
    \end{itemize}
  \end{itemize}
\item
  In der Praxis passiert oft, dass Leute \textbf{illegal} (= ohne
  Papiere) nach DE kommen und nur später das Asylverfahren gestartet
  wird

  \begin{itemize}
  \tightlist
  \item
    Aufgaben der Polizei, wenn sie erwischt werden / wenn sie sich an
    die Behörden anwenden:

    \begin{enumerate}
    \tightlist
    \item
      Strafanzeige erstatten (für illegalen Aufenthalt)
    \item
      an die zuständigen Behörden weiterleiten, d.h. Ausländerbehörde
      kontaktieren
    \end{enumerate}
  \end{itemize}
\item
  Asylbewerber haben \textbf{Meldepflicht} = sich in einer
  Erstaufnahme-Einrichtung melden

  \begin{itemize}
  \tightlist
  \item
    In anderen Fällen, kommen die Menschen mit einem Visum und wohnen
    bei Familie und Bekannten für ca. 1 Monat, nur dann beantragen sie
    Asyl\\
    → es handelt sich um einen Verstoß gegen Meldepflicht, aber passiert
    oft und werden so gut wie keine Konsequenzen gezogen

    \begin{itemize}
    \tightlist
    \item
      empfohlen: sich nach 2-3 Tage melden → beim Anhörung wirkt es
      plausibler
    \item
      auf \textbf{gelbe Briefe} achten!\\
      → es besteht keinen Pflicht, sich zu äußern, aber es wird
      empfohlen\\
      → im Fall, Kopie des Visums anlegen
    \item
      beim späteren Asylantrag kann es vorgeworfen werden, dass man
      gelogen hat und dass keine gravierende Gefahr besteht
    \end{itemize}
  \end{itemize}
\end{itemize}

\hypertarget{einreise-auf-dem-landweg}{%
\subsection{Einreise auf dem Landweg}\label{einreise-auf-dem-landweg}}

\begin{itemize}
\tightlist
\item
  gleich bei den Grenzbehörden bekannt machen, dass man Asyl sucht
\item
  ob DE zuständig ist, wird von den Grenzbehörden überprüft
\item
  über Asylantrag entscheidet BAMF
\end{itemize}

\hypertarget{einreise-auf-dem-luftweg}{%
\subsection{Einreise auf dem Luftweg}\label{einreise-auf-dem-luftweg}}

\begin{itemize}
\tightlist
\item
  Mensch ohne Papiere → im \textbf{Transitbereich} eingehalten, hier
  findet das Verfahren statt, der Mensch bleibt hier bis zur
  Entscheidung

  \begin{itemize}
  \tightlist
  \item
    \textbf{Transitbereich} ist offiziell noch nicht in DE (illegale
    Einreise)
  \end{itemize}
\item
  Mensch kann in eine Einrichtung untergebracht werden, wenn Verfahren
  länger dauert → Entscheidung muss von BAMF getroffen werden
\end{itemize}

\hypertarget{identituxe4tsfeststellung}{%
\subsection{Identitätsfeststellung}\label{identituxe4tsfeststellung}}

\begin{itemize}
\tightlist
\item
  Vorbereitung zur Antragsstellung
\item
  geschieht durch Vorzeigen von den Papieren, die man hat
\item
  bei illegaler Einreise = \textbf{Fingerabdrücke} für alle ab 6 Jahren
  (vollendet) → können aber später nachgefragt werden\\
  \textless{} 6 Jahre = nur \textbf{Lichtbilder}\\
  → wenn es passiert, ist es rechtswidrig → man kann verlangen, dass die
  Fingerabdrücke gelöscht werden\\
  - Grenzfall, wenn das Kind \textless{} 6 Jahre ist aber in der
  Zwischenzeit hat er Geburtstag → Richter kann entscheiden, dass
  Fingerabdrücke erhalten bleiben, auch wenn deren Sammlung rechtswidrig
  war
\item
  \textbf{Videoaufzeichnungen} nicht erlaubt (kein Mehrwert für die
  Identitätsprüfung ≠ Fingerabdrücke, Iris = unveränderbar)
\item
  \textbf{Sprachaufzeichnungen} erlaubt, für die Bestimmung der
  Herkunftsregion

  \begin{itemize}
  \tightlist
  \item
    überprüfen, dass die Angaben richtig sind → nach Sprachgebrauch kann
    man durch sprachwissenschaftliche Methoden Herkunft feststellen
    (bes. wichtig, wenn Sprache in vielen Ländern verbreitet ist, z.B.
    Arabisch)

    \begin{itemize}
    \tightlist
    \item
      Sprachwissenschaftliche Gutachten sind aus Eritrea nicht machbar →
      genauer Herkunftsregion kann nicht festgestellt werden
      (Sprachgebrauch ist identisch)
    \end{itemize}
  \item
    können auf Anfrage von Sprachmittlern beantragt werden → BAMF

    \begin{itemize}
    \tightlist
    \item
      Menschen können auch später zur Sprachaufzeichnung aufgefordert
      werden, wenn es Zweifeln über tatsächliche Herkunft bestehen

      \begin{itemize}
      \tightlist
      \item
        wenn er nicht zum Termin erscheint = als Desinteresse
        interpretiert
      \end{itemize}
    \end{itemize}
  \end{itemize}
\end{itemize}

\hypertarget{antragstellung}{%
\subsection{Antragstellung}\label{antragstellung}}

\begin{itemize}
\tightlist
\item
  Bei der \textbf{Anhörung} können landespezifische Fragen gestellt
  werden, z.B.

  \begin{itemize}
  \tightlist
  \item
    Name des Regierungschefs / Presidents
  \item
    Wann sind wichtige Feiertage, typische Gerichte
  \item
    Name der Straße, wo man gelebt hat; Stadtteil aufzeichnen
  \item
    Plätze beschreiben
  \item
    wie viel ein Kilo Fleisch kostet\ldots{}
  \end{itemize}
\item
  normalerweise Antragsstellung erfolgt \textbf{persönlich}, kann aber
  auch \textbf{schriftlich} stattfinden, z.B.

  \begin{itemize}
  \tightlist
  \item
    Studium (wenn man angefangen hat und zu Ende studieren möchte)
  \item
    Krankheitsaufenthalt
  \item
    Anmeldung eines Kindes bei der Geburt
  \end{itemize}
\item
  \textbf{Verlängerung / Ablehnung} des Asylstatus wird von der
  Ausländerbehörde durchgeführt

  \begin{itemize}
  \tightlist
  \item
    Ausländerbehörde entscheidet nicht → Angaben sind für die Behörde
    Online abrufbar (z.B. steht: Verlängerung für 3 Monate, 1 Jahr,
    keine Verlängerung\ldots)
  \item
    Ausnahme wenn Asyl nicht mehr nötig ist, z.B. durch Heirat mit
    DE-Bürgern → man bekommt 3-jähriges Aufenthaltserlaubnis für
    Familienzusammenführung → Voraussetzungen:

    \begin{enumerate}
    \tightlist
    \item
      Eheschließung
    \item
      Zusammenleben
    \end{enumerate}
  \end{itemize}
\end{itemize}

\hypertarget{ausfuxfcllung-des-asylantrags-durch-krankenhaus}{%
\subsubsection{Ausfüllung des Asylantrags durch
Krankenhaus}\label{ausfuxfcllung-des-asylantrags-durch-krankenhaus}}

→ damit die Versorgungskosten vom Staat gedeckt werden\\
\includegraphics{C:/Users/Admin/Alida LYT/Pasted image 20230606150315.png}

\hypertarget{mitwirkungspflicht}{%
\subsubsection{Mitwirkungspflicht}\label{mitwirkungspflicht}}

\begin{itemize}
\tightlist
\item
  die Asylbewerber bekommen \textbf{Merkblätter} auf Deutsch und in
  einer bekannten Zielsprache
\item
  \textbf{Dokumente} können abgenommen werden → normalerweise zur
  Ausländerbehörde zur Echtheitsprüfung → \ul{Quittung}

  \begin{itemize}
  \tightlist
  \item
    Wenn man die Dokumente braucht, aber die Quittung nicht gefunden
    werden kann → Anwalt zur Besichtigung der Akten beantragen und nach
    der Quittung suchen
  \item
    \textbf{Echtheitsprüfung} (PTU) durch:

    \begin{enumerate}
    \tightlist
    \item
      Untersuchung nach nachträgliche Änderungen → angemerkt welche und
      wo
    \item
      Echtheit der Urkunde → Vergleich mit Blanko-Urkunden,
      Siegelausdrücken, etc.
    \end{enumerate}
  \end{itemize}
\item
  in einer Aufnahmeeinrichtung wohnen
\item
  persönliche Meldepflicht beim Aufruf
\item
  bei Ermittlung des Sachverhalts (z.B. Sprachgutachten) → die Person
  muss erscheinen!
\item
  \textbf{Postempfang} muss gesichert werden → wenn die Person umzieht,
  muss sie die neue Adresse abgeben → Zustellung ist wirksam, auch wenn
  die Person die Briefe nicht bekommt

  \begin{itemize}
  \tightlist
  \item
    Behörde verwenden die letztbekannte Adresse für die Zustellung
  \end{itemize}
\end{itemize}

\hypertarget{zustuxe4ndigkeit}{%
\subsection{Zuständigkeit}\label{zustuxe4ndigkeit}}

\begin{itemize}
\tightlist
\item
  unbegleitete \textbf{Minderjährige}

  \begin{itemize}
  \tightlist
  \item
    Land des letzten Asylantrages
  \item
    es wird immer zur Wohle des Kindes entschieden → z.B. wenn
    Familienangehörige sich in Schweden befinden, wird Schweden
    zuständig
  \end{itemize}
\item
  \textbf{begünstigter internationaler Schutz} (Flüchtlingsstatus /
  subsidiärer Schutz)

  \begin{itemize}
  \tightlist
  \item
    z.B. Familienangehörige (= Ehepaar und Kinder, keine andere
    Verwandten) haben bereits internationalen Schutz in einem Land
    bekommen, ist das gleichen Land für alle Familienangehörige
    zuständig
  \item
    muss schriftliche Bewilligung von beiden Eheleuten bestehen
  \end{itemize}
\item
  \textbf{internationalen Schutz beantragt} → ähnlich wie oben. Auch
  hier muss eine schriftliche Bewilligung vorliegen
\item
  \textbf{Verschuldensprinzip} = wenn ein Visum besteht, ist das Land,
  das das Visum erstellt hat, zuständig

  \begin{itemize}
  \tightlist
  \item
    nach dem Prinzip: Die Person hätte nach DE nicht kommen können, wenn
    sie kein Visum bekommen hätte
  \item
    Visum für andere Zwecken (nicht für Asyl): medizinische Versorgung,
    Tourismus, Betriebsreise...
  \end{itemize}
\end{itemize}

\hypertarget{uxfcbungsfuxe4lle}{%
\section{Übungsfälle}\label{uxfcbungsfuxe4lle}}

Bestimmung der Zuständigkeit\\
\includegraphics{C:/Users/Admin/Alida LYT/Pasted image 20230606153327.png}

\begin{enumerate}
\tightlist
\item
  Nach Verschuldensprinzip ist Polen zuständig, da das Land ein
  Schengen-Visum erstellt hat
\item
  für Minderjährige, letztes Land, wo Asyl beantragt wurde\\
  → wenn Asyl schon in Italien beantragt, BAMF teilt italienischen
  Behörden mit, dass Zweitantrag gestellt wurde\\
  → wenn über Asyl schon entschieden und zugestimmt (z.B.
  Flüchtlingsstatus anerkannt), kann nur für kurze Zeit nach DE
  einreisen\\
  - Abschiebungsverbot nach Italien soll geprüft werden (aus Erfahrung,
  pauschal abgelehnt)\\
  - unter triftigen Gründen erlaubt, z.B. Kind besucht schon Schule in
  DE, oder in Italien ist es alleine, während in DE gibt es Verwandten,
  bei den es sich wohl fühlt

  \begin{itemize}
  \tightlist
  \item
    Immer zur Wohl des Kindes entschieden
  \end{itemize}
\item
  begünstigter internationaler Schutz in einem EU-Land
\end{enumerate}

\end{document}
